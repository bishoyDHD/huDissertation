\chapter{Introduction}
%\section{Introduction}
\label{labint}
 The TREK/E36 experiment was conducted at the Japan Proton Accelerator Research Complex (J-PARC), with physics data-taking occurring from September to December 2015, and has been decommissioned as of March 2016. TREK/E36 was part of the TREK program at J-PARC, where TREK stands for Time Reversal Experiment with Stopped Kaons. The primary goal of the TREK/E36 experiment was to provide a high precision electroweak measurement in order to test lepton universality, which is expressed as an identical coupling constant of the charged lepton family $(e,$ $\mu,$ and $\tau)$. Lepton universality is a staple of the Standard Model (SM) and any violation of this would be clear evidence of New Physics (NP) beyond the SM. 
 
 Data analysis of two-body leptonic decays of stopped kaons $K_{l2}$ is currently being performed in order to provide a precise measurement of the decay width ratio $R_K=\Gamma(K^+\rightarrow e^+\nu)/\Gamma(K^+\rightarrow\mu^+\nu)$ to test lepton universality. Due to the $V-A$ structure of charged current couplings, the ratio $R_K$ is helicity-suppressed in the SM and is therefore sensitive to beyond SM physics. The $K_{l2}$ decay width is calculated as
 
 \begin{equation}
  \Gamma(K_{l2})=g^{2}_{l}\frac{G^2}{8\pi}f^{2}_{K}m_Km^{2}_{l}\left(1-\frac{m^{2}_{l}}{m^{2}_{K}}\right)^2
  \label{eqn:dewidth}
 \end{equation}

 \noindent where $g_l$ is the coupling constant of the lepton current, $G$ is the Fermi constant, $f_K$ is the kaon form factor and $m_K$ and $m_l$ are the kaon and lepton masses respectively. The SM value for $R_K$ is very precise because to a first approximation the strong interaction dynamics from equation \ref{eqn:dewidth} cancel
 
 \begin{align}
  R^{SM}_{K}&=\frac{\Gamma(K^+\rightarrow e^+\nu)}{\Gamma(K^+\rightarrow\mu^+\nu)}\nonumber \\
  &=\frac{m^{2}_{e}}{m^{2}_{\mu}}\left(\frac{m^{2}_{K}-m^{2}_{e}}{m^{2}_{K}-m^{2}_{\mu}}\right)^2(1+\delta_r)\nonumber \\
  &=(2.477\pm0.001)\times10^{-5}
  \label{eqn:ratioRk}
 \end{align}

 where $\delta_r$ represents radiative corrections, detailed calculations of which were carried out in. Thus the SM value for $R^{SM}_{K}$ has been calculated to high accuracy $(\Delta R_K/R_K\sim0.4\times10^{-4})$ thereby making it possible to search for NP effects by conducting a precise measurement of $R_K$.
