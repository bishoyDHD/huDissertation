\begin{abstract}
 The Standard Model (SM) represents our best description of the subatomic world and has been very successful in explaining how elementary particles interact under the influence of the fundamental forces. Despite its far reaching success in describing the building blocks of matter, the SM is still incomplete; falling short to explain dark matter, baryogenesis, neutrino masses and much more. The E36 experiment conducted at J-PARC in Japan, allows for sensitivity to search for light $U(1)$ gauge bosons, in the muonic $K^+$ decay channel. Such $U(1)$ bosons could be associated with dark matter or explain established muon-related anomalies such as the muon $g-2$ value, and perhaps the proton radius puzzle. A realistic simulation study was employed for these rare searches in a mass range of 20 MeV to 100 MeV. Currently, about $\sim8\%$ of the data has been analyzed and upper limits for the $A^\prime$ branching ratio $\mathcal{B}r(A^\prime)$ have been extracted at 95\% CL.
 
\end{abstract}
